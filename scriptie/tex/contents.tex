\documentclass[MScProj_TLRH_ClusterEnergy.tex]{subfiles}

\begin{document}
\chapter*{Thesis Outline}
\label{sec:outline}
Hierarchical clustering is the current cosmological paradigm explaining the
formation of large-scale structure in the universe. Smaller structures such as 
stars and galaxies are believed to have formed first, followed by larger 
structures clumping together in continuous merger events. Cluster of galaxies 
are the largest bound structures in the universe at present. Within this 
framework, the presence of hot gas in the intra cluster medium (ICM) is 
explained naturally, as is the occurrence of mergers of these objects. X-ray 
observations of clusters show that the ICM radiates thermal bremsstrahlung, a 
radiative loss that cools the ICM. Order of magnitude timescale arguments show 
that these radiative losses would cool the ICM on a timescale significantly 
shorter than one Hubble time. It thus remains a question how the ICM sustains 
the X-ray emission on cosmological timescales as it requires a source of heating.
Two possible processes that could inject thermal energy into the ICM are active
galactic nucleii (AGN), and cluster-cluster mergers. As such, the ICM could be
reheated and clusters of galaxies would continue to emit the X-rays observed at
present.

Cygnus~A is a unique astronomical object as it hosts an active galactic nucleus
inside its brightest cluster galaxy (BCG) while the cluster, at the larger scale, is 
currently in the early stage a major merger (Figure~\ref{fig:CygA_Xray_extended}
in chapter~\ref{sec:constrains}). Therefore it is of no surprise that this object 
has been subject to numerous studies over the last decades. As of July 2016, 
querying Simbad for references on Cygnus~A yields no less than 1795 publications
that mention the source. Cygnus~A is an excellent candidate to study the effects
of a major merger on the thermal structure of the intra cluster medium. Moreover,
the relative contribution of the AGN and the merger to (re-)heating the ICM can be 
studied in the same object.

The optical cD galaxy Cygnus~A, when observed in the radio (Figure~\ref{fig:CygA_Radio}
in section~\ref{sec:cygnus}), shows two jets and two lobes associated with the AGN.
The collimated outflow results from accretion of the surrounding gas onto a 
supermassive black hole (SMBH) located at the galactic center. Simultaneously, 
the jets sweep up gas surrounding the galaxy, thus powering the two radio lobes 
starting at a distance of \mytilde30~kpc from the galaxy, and creating hot spots 
at the jet termini. This process is called AGN feedback because the very gas 
reservoir that powers the AGN itself is altered as the outflow displaces and 
heats the gas. The latter significantly influences the star formation rate of the
host galaxy as well as the properties cluster environment. The unique radio 
signature of Cygnus~A make this source the prime example of an active galactic 
nucleus interacting with its environment (Figure~\ref{fig:CygA_Xray} in 
chapter~\ref{sec:constrains}). In fact, Cygnus~A is the archetype of the
\citet{1974MNRAS.167P..31F} type II radio galaxy classification, abbreviated as FR-II.

Observations have shown that the X-ray emission of the ICM associated with the 
Cygnus~A cluster peaks at the exact same coordinates on the sky at which the radio
emission from the AGN in the Cygnus~A galaxy is found. This makes the X-ray photons
originating from the ICM an excellent tool to probe the interaction of Cygnus~A 
with its environment. The relative proximity of Cygnus~A at redshift $z = 0.0562$ 
allows for AGN~feedback studies with unprecedented spatial resolution. In addition,
the strong radio emission is a thousand times brighter than any other source at 
the same distance, as can be seen in Figure~\ref{fig:CygA_Radio_bright}. In general,
AGN at lower redshifts show lower average radio luminosities. Cygnus~A, on the
other hand, is the exception that hosts an AGN equivalent in power to an active 
galactic nucleus at a redshift of $z=1$. The high luminosity observed from the 
radio lobes make Cygnus~A the brightest extragalactic radio source in the sky.
This combination of proximity and extreme brightness shows the energy transport
in the innermost kpc surrounding the SMBH and make Cygnus~A the ideal target to
study AGN~feedback in detail. In a broader context, understanding the fine details
and inner workings of AGN~feedback is crucial to understand structure formation 
and evolution in the universe as a whole. Furthermore, AGN~feedback affects the 
growth of the very black hole that powers the AGN in the first place. The study 
of AGN~feedback could further our understanding of how black holes grow and evolve. 

In this thesis we focus on the diffuse X-ray emission at the megaparsec scale 
that originates from the ICM of the Cygnus~A cluster of galaxies. In particular,
we infer initial conditions to set up numerical simulations of the merger from
recent X-ray observations, followed by results from the simulations.
We are interested in whether or not the merger could reheat the ICM, and if so 
where and how much energy is thermalised. Furthermore, we want to compare this
energy injection to the amount of energy injected by the AGN to find the relative
importance of heating the ICM of both processes. With these goals in mind we 
investigate the following questions. What are the masses of the two merging sub 
clusters in the Cygnus system? What is the velocity at which both clusters are
moving towards each other? How long before core passage is the merger at present?

To address these questions we start with a very general introduction into the 
current cosmological paradigm and work our way up to large-scale structure 
formation in section~\ref{sec:structure}. We discuss the distribution of dark 
matter in clusters of galaxies in section~\ref{sec:darkmatter}, followed by 
the general properties of clusters associated with the X-ray emission in 
section~\ref{sec:clustergas}. We continue with a summary of the status quo of 
the literature on observations of Cygnus~A in section~\ref{sec:cygnus}. 
Note that due to the humongous collection of such publications we have to limit
our scope to X-ray observations of the ICM and merger-related publications only, 
but we do include optical studies that are relevant to the merger. However,
no literature on the radio source is included in the overview. Next, we discuss 
how we have used constrains from recent X-ray observations to infer initial 
conditions to set up numerical simulations of the merging cluster of galaxies in
chapter~\ref{sec:constrains}. A detailed description of the numerical setup and 
methods is given in section~\ref{sec:methods}, followed by the simulation results 
in section~\ref{sec:sim}, discussed in section~\ref{sec:discussion} to come 
to our conclusions in section~\ref{sec:conclusions}.



\pdfbookmark[1]{\contentsname}{tableofcontents}
\setcounter{tocdepth}{2} % <-- 2 includes up to subsections in the ToC
\setcounter{secnumdepth}{3} % <-- 3 numbers up to subsubsections

\tableofcontents

\SubfileBibliography
\end{document}
