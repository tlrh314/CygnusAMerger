% File: tex/abstract.tex
% Author: Timo L. R. Halbesma <timo.halbesma@student.uva.nl>
% Version: 0.01 (Initial)
% Date created: Sat Oct 24, 2015 12:28 am
% Last modified: Mon Sep 05, 2016 07:21 PM
%
% Description: Masterthesis, abstract

\documentclass[MScProj_TLRH_ClusterEnergy.tex]{subfiles}
\begin{document}
\pdfbookmark[1]{Abstract}{Abstract}
\chapter*{Abstract}


The nearby Cygnus~A galaxy (z = 0.0562) hosts an Active Galactic Nucleus (AGN) 
that is a thousand times brighter than any other AGN at the same distance and
is the brightest extragalactic source of radio emission. This source is the 
archetype of the FR-II class of classical radio doubles interacting with its 
environment. At the same time, the galaxy is part of a moderately rich cluster
that is about to undergo a major merger between two sub-clusters about 0.5 Gyr
prior to core passage. In this thesis the deepest data set of any merging cluster 
of galaxies to date is combined with a numerical representation of the system 
generated using the state-of-the art proprietary initial conditions code
\code{Toycluster} to simulate the cluster merger. The exposure time at present
is already an impressive mega second of \satellite{Chandra}~XVP, while the 
final total combined exposure time will reach 2.2~mega seconds within the coming
year. Spherically symmetric haloes are sampled using the Hernquist profile, an 
analytical expression equal to the NFW-profile within a scaling radius. This 
governs the dark matter content, while the baryonic matter has a fixed baryon 
fraction of seventeen per cent at the virial radius and is assumed in 
hydrostatic equilibrium. The betamodel then describes the density structure of the
hot intra cluster medium, and the pre- and post-merger density and temperature
profiles are simulated using the massively parallel publicly available TreeSPH 
code \code{Gadget-2}. We investigate the effect of the initial velocity and the
projection angle on the line-of-sight integrated two-dimensional post-simulation
projected radial profiles obtained using the proprietary map making tool
\code{P-Smac2}. Our analytical models estimate the total gravitating mass and the 
concentration parameter of the dark matter halo, both of which can be 
observationally tested in a weak-lensing study. No such study has been published 
to date, to our knowledge. Our simulations neglect the AGN-activity, thus provide
both the hydrostatic and the merger temperature structure which can be 
subtracted from the measured temperature profiles. The remaining temperature 
structure then most likely arises as a result of heating the intra cluster medium
surrounding the central galaxy Cygnus~A by the powerful AGN excluded from our
numerical models. This system uniquely allows us to study the complicated process
of AGN~feedback with a spatial resolution and a level of detail unprecedented in
prior studies. For example, the duty cycle and AGN outburst history of the last 
few hundred mega years could be inferred from the residual temperature maps, which
promises exciting results in the foreseeable future. 


%\SubfileBibliography
\end{document}
