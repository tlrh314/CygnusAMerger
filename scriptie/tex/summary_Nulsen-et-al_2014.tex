% File: tex/99_summary_Nulsen-et-al_2014.tex
% Author: Timo L. R. Halbesma <timo.halbesma@student.uva.nl>
% Version: 0.01 (Initial)
% Date created: Sat Oct 24, 2015 12:28 am
% Last modified: Mon Jan 25, 2016 03:14 PM
%
% Description: Literature summary

\documentclass[MScProj_TLRH_ClusterEnergy.tex]{subfiles}
% Allow to compile this document on itself using the preamble of ../MScProj_TLRH_ClusterEnergy.tex

\begin{document}
\section*{Interaction of Cygnus A with its environment}
\label{sec:nulsen2014}
Summary of \citep{2015IAUS..313..236N}. NB, this is a conference proceeding.
\\
\textbf{Abstract}
The closest powerful radio galaxy Cygnus A emits both radio and X-ray emission. The galaxy is classified as a Faranoff-Riley class II and lives at a redshift $z=0.056$. In particular, Chandra observations have shown X-ray cavities in the radio lobes, X-ray jets, and cocoon shocks. The emission mechanism for the shock is thought to be non-thermal, while the X-ray jets are caused by synchrotron radiation. The latter implies that the X-ray jets,rather than the radio jets, could be injecting energy from the central supermassive black hole into the surrounding medium causing visible hotspots. Assuming that the energy flow is indeed governed by the X-ray jets it is found that these jets inject about 1 M\Sun yr\Sup{-1} are non-relativistic.
\\
\textbf{Introduction}
The FR-II radio galaxy Cygnus A living at $z = 0.056$ is the central galaxy of the Cygnus A cluster. X-ray images show interactions between the expanding radio lobes and the intracluster medium. In this article properties of Cygnus A inferred from \emph{Chandra} observations are discussed.
\\
\textbf{Shocks and Lobes}
A reduced 246 ksec \emph{Chandra} image in the 0.5 - 7 keV energy range observed in the 2000-2005 period is shown in Fig.~\ref{fig:CygA_Chandra}. The contour overlay shows 0.001, 0.003, 0.01, 0.03, 0.1 Jy beam\Sup{−1} 6 cm radio emission. In this image 1 arcsec corresponds to 1.1 kpc.
\\
The X-ray emission is brighter where the radio lobes end because the gas is compressed and heated, thus, is brighter than undisturbed gas and produces an excess of X-ray emission over the radio cocoon. This proffers that X-ray cavities are caused by radio lobes.

\begin{figure}[h!]
\centering
\includegraphics[scale=0.6]{Nulsen2014_Chandra246kec_2000-2005_Unedited.png}
\caption{Reduced 246 ksec \emph{Chandra} observation taken between 2000-2005 with 6 cm radio contour overlay from \citet{1984ApJ...285L..35P}. Image adopted from \citep{2015IAUS..313..236N}.}
\label{fig:CygA_Chandra}
\end{figure}

\textbf{Mean jet power}
The average outburst power of Cygnus A can be inferred from the Cocoon shocks.To do so a spherical, numerical fluid shock model is used to calculate the X-ray surface brightness of the shock front. The initial conditions consist of an isothermal atmosphere in hydrostatic equilibrium (see Nulsen et. al 2005). It is found that the unperturbed gas density depends on $r$\Sup{-1.38}, the shock radius is 40 kpc, the Mach number is 1.37, the age 16 Myr, and the mean power is 4 $\times$ 10\Sup{45} erg s\Sup{-1}. There are some over- and under estimates, but the average jet power should be accurate within a factor of two.

\textbf{Particle acceleration in the shock}
The inner region of the shock could be thermal, while the outer parts could be nonthermal based on similarities with Centaurus A (see e.g. Creston et. al 2009 for the Centaurus A article). Because the radio emission is confined in the region showing X-ray emission it must arise from shocked gas, thus, these regions are sites of particle acceleration. The shocks fronts near the X-ray hotspots are estimated to have Mach number $>$3 and are too bright to be thermal.

\textbf{X-ray jet}
There is an X-ray jet that is slightly misaligned with the radio jet and it is wider and (6 kpc / 6 arcsec) and it is straighter. Steenbrugge et al. (2008) conclude claim that relative low energy electrons injected by an earlier radio jet produce the X-ray jet via inverse Compton emission. Doppler boosting is unlikely to be the emission mechanism, and IC scattering of the CMB is also ruled out, according to Nulsen et al. (2014). On the other hand, synchrotron emission could explain the characteristics of the X-ray jet, if a power source is present to provide enough relativistic electrons.

\textbf{Conclusions}
The mean output power is 4 $\times$ 10\Sup{45} erg s\Sup{-1}. The emission mechanism of the X-ray jet is most likely synchrotron emission, thus, the energy emitted by the AGN follows the X-ray jets rather than the radio jets.

\SubfileBibliography
\end{document}
