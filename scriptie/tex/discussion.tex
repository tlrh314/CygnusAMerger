% File: tex/discussion.tex
% Author: Timo L. R. Halbesma <timo.halbesma@student.uva.nl>
% Version: 0.01 (Initial)
% Date created: Sat Oct 24, 2015 12:28 am
% Last modified: Fri Aug 12, 2016 01:58 PM
%
% Description: Masterthesis, Discussion


\documentclass[MScProj_TLRH_ClusterEnergy.tex]{subfiles}
\begin{document}
\chapter{Discussion}
\label{sec:discussion}

% TODO: \citet{1978A&A....70..677C} give that
% \begin{align}
%     r_c &= 3 \left[ \frac{<v_{\parallel}^2>_{M}}{4 \pi G \rho_{G0}} \right]^{1/2} \quad ,
% \end{align}
% where $<v_{\parallel}^2>$ is the velocity
% dispersion of the galaxies parallel to the line of sight, and $\rho_{G0}$ is the 
% galaxy density (??)
% 
% For example, $r_c = 250$  for the Coma Cluster. 
% I believe this is derived from optical considerations
% assuming the optical galaxies as well as the X-ray gas follows the potential well.
% So if we take the optical Cygnus~A paper, then we should perhaps be able to infer the core 
% radius of the cluster from the optical velocity dispersion measurements?
% 
% TODO: beta = 2/3
% 
% TODO: However, these numbers do not contain the systematic error. For instance,
% for the haloes spherical symmetry is assumed. In particular for the dark matter
% halo reality could be more complicated. The dark matter haloes are more likely
% triaxial and elongated along a given axis. Estimates on the amount of elongation
% varies, but could very well be a factor of two, which would imply that our dark
% matter mass estimates have a systematic uncertainty of at least a factor four.
% 
% TODO: the baryon fraction at $r_{500}$ should be 14 \% (Planelles+ 2013).
% We, however, infer different numbers (higher baryon fractions). This would
% imply that we should perhaps calculate the concentration parameter using a
% different function than given by Duffy+ 2008.
% 
% 
% TODO: triaxiality of the DM haloes.
% 
% 
% TODO: ``Depending on the sample selection, 30 to 75 \% of clusters have substructure (Burns+ 1994 423 94, Jones+ 1994 107 1637)''
% 
% 
% 
% 
% We acknowledge a fourth model that could provide an even more accurate
% description of the observed data. A double beta model could perhaps better
% constrain the increased density observed in the inner three kpc of Cygnus~A.
% We show the raw density profile including the innermost bins
% in Figure~\ref{fig:raw_data} for completeness, but in the further analysis
% we discard the first few data points corresponding to this region. We justify this
% because, firstly, the data points have rather large error bars, thus, will have no
% significant effect on the weighted least squares fit as the data points are weighted
% with this large uncertainty, and secondly, we assume this feature arises as a result
% of AGN activity in the central galaxy of the cluster but not from the diffuse intra
% cluster gas component that we intent to constrain. Thirdly, the CCD suffers from an
% effect known as pile-up. The source in the inner region is so bright that two photons
% could hit the detector chip within the read-out time of 3.2 seconds. Pile-up is a
% serious issue for trustworthiness of the observation. Moreover, the scale at which this
% feature arises is around the same order of magnitude as the mean free path of the photons
% so we worry that the fluid approximation breaks down at this scale length. Finally,
% should a double beta model provide the best-fit it is not possible to model this feature
% in a stable manner. Scales well below the typical smoothing lengths fall within the SPH
% smoothing kernel, which means that the density is not smoothed over the full kernel radius
% and can not be trusted. This numerical limitation might be overcome by significantly
% increasing the number of particles in the inner region of three kpc, but as the particles
% are sampled over three decades the total number of particles should be increased a lot
% to obtain the required increase in the inner region. The resulting increased compute time
% is likely not worth the results it might produce, certainly not given the other arguments
% to discard the first bins, and we reject a double beta model. TODO: rephrase last argument.


\SubfileBibliography
\end{document}
