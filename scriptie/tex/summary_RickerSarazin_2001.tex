% File: tex/99_summary_RickerSarazin_2001.tex
% Author: Timo L. R. Halbesma <timo.halbesma@student.uva.nl>
% Version: 0.01 (Initial)
% Date created: Mon Nov 09, 2015 08:55 AM
% Last modified: Mon Nov 09, 2015 10:02 AM
%
% Description: Literature Summary

\documentclass[MScProj_TLRH_ClusterEnergy.tex]{subfiles}
% Allow to compile this document on itself using the preamble of ../MScProj_TLRH_ClusterEnergy.tex

\begin{document}
\section*{Off-axis Cluster Mergers: Effects of a Strongly Peaked Dark Matter Profile}
\label{sec:SarazinRicker2001}
Summary of \citet{2001ApJ...561..621R}.
\\
\section*{Abstract}
Offset mergers are simulated using COSMOS, an Eularian hydrodynamics/N-body code. Clusters have nonisothermal, hydrostatic initial conditions with a $\beta$-model gas profile and a steep dark matter profile. With use of statistical relationships obtained from observations it is possible to constrain global cluster properties of the simulated clusters. The mass ratio of merger is assumed 1:3 or 1:1, and the impact parameter is 0-5 times the dark matter scale radius. Agreement is found with previous studies using the King profile for the temperature jumps, changes in morphology and the relative velocities. On the other hand, a 4-10 times larger X-ray luminosity jump is found. The authors argue this is due to lower spatial resolution.

Constraints on $\Omega$ obtained from hot clusters at high $z$ are influenced by the temperature- and luminosity jumps due to mergers. Shocks increase entropy in the outer regions as the core is only weakly shocked. The gas from outer regions is high in entropy and will mix with core gas in later stages of the merger. This results in mixing due to ram pressure and yields convective, unstable, displaced core gas. As a result of the convection, the core gas creates turbulence on larges scales with eddies up to hundreds of kilo parsecs. Support against gravity (5-10\%) is provided by persistent subsonic turbulence in the cluster core as the turbulence is pumped by dark matter oscillations in the gravitational potential. Moreover, these DM oscillations result in large timescales to reach virial equilibrium.

% TODO: what is the dark matter scale radius?
% TODO: what is an Eularian hydrodynamics code?
% TODO: read again what the beta-profile is.



\SubfileBibliography
\end{document}
