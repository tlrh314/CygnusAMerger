% File: tex/popular_abstract.tex
% Author: Timo L. R. Halbesma <timo.halbesma@student.uva.nl>
% Version: 0.01 (Initial)
% Date created: Sat Oct 24, 2015 12:28 am
% Last modified: Mon Sep 05, 2016 07:20 PM
%
% Description: Masterthesis, abstract Dutch 


\documentclass[MScProj_TLRH_ClusterEnergy.tex]{subfiles}
\begin{document}
\chapter*{Populaire samenvatting}
\pdfbookmark[1]{Dutch Abstract}{Dutch Abstract}


Hi\"erarchische clustering is een model voor het de structuren in het Universum
die hedendaags waargenomen kunnen worden zijn ontstaan door gravitationele
instabiliteit van fluctuaties in de dichtheid in een vroeg stadium van het
Universum. Er wordt gedacht dat de sterren, melkwegstelsels en clusters van
melkwegstelsels die waargenomen kunnen worden zijn ontstaan in volgorde van
klein naar groot en dat structuren groter kunnen worden door botsingen. Clusters
van sterrenstelsels zijn hedendaags de grootste structuren die gravitationeel
gebonden zijn. Er bestaan tal van systemen van botsende clusters van
sterrenstelsels.


Cygnus~A is een voorbeeld van een cluster van sterrenstelsels waarin momenteel
een botsing van twee subclusters gaande is. Naast deze botsing op schaal van
enkele megaparsecs is het centrale sterrenstelsel van het cluster, ook Cygnus~A
genaamd, een van de laatst overgebleven krachtige radiostralers relatief
dichtbij het Zonnestelsel. Het centrale superzware zwarte gat in het centrum van
dit sterrenstelsel accreteert materie uit haar omgeving waardoor relativistische
jets tot buiten het sterrenstelsel reiken (een Active Galactic Nucleus genaamd)
en interacteren met het intra cluster medium op schaal van enkele kiloparsecs
(AGN feedback).

In dit masterproject ligt de focus op de botsing in het cluster van
sterrenstelsels Cygnus~A. Ongeveer negentig procent van de baryonische massa van
het cluster bevindt zich in het intra cluster gas. Het merendeel van de totale
massa is echter niet de baryonische materie: ongeveer vijfentachtig procent van
de totale massa bestaat uit donkere materie. De botsing van Cygnus~A wordt
numeriek gesimuleerd door smoothed particle hydrodynamics deeltjes van een
ideaal gas met een beta (2/3) model in een Hernquist donkere materie halo te
plaatsen. Hierbij negeren we stervorming, koudestromen, gebruiken we concordance
cosomology, en simuleren we ge\"idealiseerde \'e\'en-op\'e\'en botsingen in plaats van volledige cosmologische simulaties te runnen. Het voordeel hiervan is dat er totale controle over de parameters van de botsing is.


Het intra cluster gas heeft een temperatuur van enkele miljoenen Kelvin dus
zendt thermische Brehmsstrahlung uit met r\"ontgen golflengtes. De begin
conditie van de simulatie hangt af van de parameters van het gas, die verkregen
kunnen worden met een diepe (\mytilde 1 Msec) r\"ontgen observatie van de
\satellite{Chandra} satelliet. De botsing van de subclusters is ongeveer 500
megajaar voordat de centrale regio's bij elkaar komen, dus de waarneming levert
vrijwel rechtstreeks de begincondities. Zodra de begincondities verkregen zijn
wordt de simulatie gestart met de publiek beschikbare cosmologische
simulatiecode \code{Gadget-2}. Hier wordt de zwaartekracht van de deeltjes
gecombineerd met de hydrodynamica om de staat van het gas te modelleren. De
driedimensionale simulatie wordt vervolgens geprojecteerd op een twee
dimensionaal vlak waarna fysische grootheden berekend kunnen worden, zoals de
r\"ontgen emissiviteit. Hiermee is een synthetische observatie vergaard die
vergeleken kan worden met de \satellite{Chandra} waarneming om vervolgens de
begincondities te tweaken. Met deze methode wordt direct een voorspelling gedaan
van de massa van de subclusters, de hoeveelheid en verdeling van de donkere
materie binnen het cluster, en kunnen andere, nog niet waargenomen
golflengtegebieden berekend worden.


TODO: voeg toe wat de resultaten zijn, discussies en conclusies.
TODO: waarschijnlijk is het noodzakelijk iets minder verbose te zijn ...


%\SubfileBibliography
\end{document}
